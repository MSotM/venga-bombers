\section{Rendering}
\label{sec:rendering}

The game supports rendering to multiple devices. First and foremost the game is
rendered to a Watterott MI0283QT9 TFT-LCD display. Second, for development
purposes, the game can be rendered to a serial monitor using ANSI escape codes.
Not every developer might have a display available, but all of them have a
terminal to use a serial monitor. The method of rendering and support of
multiple displays is explained in this section.

\subsection{Incremental rendering}
\label{sec:incremental-rendering}
Due to the limitations of the LCD display, namely that rendering the entire
screen takes a good half a second or so, the game has to be rendered in
increments. Since the game is based on a grid (See Section~\ref{sec:world}) we
can keep track of which parts of the grid have been updated and thus need to be
rendered. The rest of the field is already drawn and can be left alone. This
way we still achieve smooth visuals, even when rendering to both a terminal and
the LCD display.

Any action that directly affects the world sets a render update flag on the
involved tiles indicating that those tiles need to be rendered. The
\texttt{render} function then loops through the tiles and checks which of
them need rendering, does so, and then resets the render update flag. The sixth
bit, one of the leftover bits available in each tile, is used for this (See
Section~\ref{sec:tiles}). This bit is accessed using the functions
\texttt{tile\_needs\_render\_update} and \texttt{tile\_set\_render\_update}.

\subsection{Multiple displays}
\label{sec:rendering-code}
The file \texttt{game.h} has function definitions for setting up rendering and
actually rendering. These are \texttt{init\_render}, \texttt{render},
\texttt{init\_lcd\_display}, \texttt{render\_to\_lcd},
\texttt{init\_terminal\_display} and \texttt{render\_to\_terminal}.

These are implemented in \texttt{render.c}, \texttt{lcd-renderer.c} and
\texttt{terminal-renderer.c} respectively. The \texttt{init\_render} and
\texttt{render} functions call the LCD and terminal rendering functions,
depending on whether \texttt{RENDER\_LCD} and/or \texttt{RENDER\_TERMINAL} are
defined in the file \texttt{options.h}.

The \texttt{render} function accepts a boolean to force rendering of all
tiles regardless of whether the render update flag has been set for any of the
tiles. This is used to render the initial playing field before the game starts.

Communication with the LCD display is done using \texttt{pleasant-lcd}, while
communication with the serial monitors happens through USART using
\texttt{pleasant-usart}, both of which are from the \texttt{pleasant-uno-avr}
library (See Section~\ref{libraries}).

