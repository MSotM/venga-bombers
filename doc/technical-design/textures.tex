\section{Textures}
\label{sec:textures}

Almost all rendering to the LCD (see Section~\ref{sec:rendering}) is done in
the form of \emph{textures}. They are used for every tile, every number, and
most of the UI\@.

Textures are encoded as arrays of 16-bit \emph{words}, stored in program
memory. These arrays consist of a 1-word \emph{header} (see
Section~\ref{sec:texture-header}), a \emph{color table} (see
Section~\ref{sec:texture-color-table}), and \emph{pixel data} (see
Section~\ref{sec:texture-pixel-data}).

\subsection{Header}
\label{sec:texture-header}

The first word of a texture definition is the \emph{header}, which encodes the
following properties of the texture:

\begin{itemize}
\item 3 bits indicating the texture size

  \begin{enumerate}
  \item \texttt{TEXTURE\_SIZE\_TILE} (0) for a texture of 16 by 16 pixels.
  \item \texttt{TEXTURE\_SIZE\_UI} (1) for a texture of 80 by 120 pixels.
  \item \texttt{TEXTURE\_SIZE\_CHARACTER} (2) for a texture of 5 by 8 pixels.
  \item \texttt{TEXTURE\_SIZE\_PLAY\_BUTTON} (3) for a texture of 80 by 30
    pixels.
  \item \texttt{TEXTURE\_SIZE\_MENU\_TITLE} (4) for a texture of 160 by 32
    pixels.
  \end{enumerate}

\item 4 bits specifying the number of colors the texture uses
\item 2 bits specifying the \emph{unit size}. See
  Section~\ref{sec:texture-pixel-data}
\item 1 bit specifying the \emph{encoding type}. See
  Section~\ref{sec:texture-pixel-data}
\end{itemize}

This leaves 6 bits unused.

\subsection{Color table}
\label{sec:texture-color-table}

Immediately after the \emph{header} starts the \emph{color table}. It is a
sequence of colors, the length of which depends on the number specified in the
header. Each color is a 16-bit number, which can directly be sent to the LCD\@.

Color numbers can be specified using the \texttt{RGB} preprocessor macro, which
converts \texttt{r}, \texttt{g} and \texttt{b} components to the closest 16-bit
color. This macro is provided by the LCD library.

\subsection{Pixel data}
\label{sec:texture-pixel-data}

All data after the color table is \emph{pixel data}. It specifies the content
of the texture's pixels from left to right and from top to bottom. The exact
meaning of the data depends on the \emph{unit size} and the \emph{encoding
  type}, both of which are specified in the header (see
Section~\ref{sec:texture-header}).

The \emph{encoding type} is either \texttt{TEXTURE\_ENCODING\_PLAIN} or
\texttt{TEXTURE\_ENCODING\_RUN\_LENGTH}. When it is \emph{plain}, each pixel of
the texture is specified using a single \emph{unit}, which serves as an index
into the \emph{color table} (See Section~\ref{sec:texture-color-table}). When
it is \emph{run-length}, the \emph{units} alternate between indicating a
run-length and indicating a color. The run-length specifies the number of
pixels that should be filled with the color that follows it.

As an example consider a texture with a total size of 5 pixels, where the first
three pixels use the first color from the color table and the last two pixels
use the second color from the color table. We will use two pixels to encode
each color.

When using \emph{plain} encoding, the relevant part of the pixel data would be
$00\;00\;00\;01\;01_2$. When using \emph{run-length} encoding, however, it
would be $11\;00\;10\;01_2$. This already saves two bits, for such a small
number of pixels.

\subsection{Texture encoder}
\label{sec:texture-encoder}

Because encoding images into the format specified above is a tedious process, a
small script is included (\texttt{scripts/generate-textures}) that takes images
in PNG format from the \texttt{textures} directory and generates the C code to
save them in a variable. The output of this script is available in the file
\texttt{src/textures.c}.

The script is not optimal---because it always generates textures with
\emph{plain} encoding---but it does make the texture encoding process painless.
