\section{Music}

One of the reasons the project name was chosen to be ``Venga Bombers'', is that
the `Venga Boys'\cite{venga-boys} have a song called ``Boom, Boom, Boom,
Boom!!''. This seemed appropriate, but only if we could somehow include the
song in the game. We have managed to play the main melody during the game.
Including more then one voice was deemed impossible due to the limited number
of timers available (see Section~\ref{sec:music-timer}). One problem we
encountered when trying to include the song is that only a limited amount of
memory is available. This was resolved by using a particular encoding for notes
(see Section~\ref{sec:music-memory}).


\includegraphics[width=\textwidth,height=\textheight,keepaspectratio]
  {res/boom.png}

\subsection{Timer Management}
\label{sec:music-timer}
We used a timer to generate tones by flipping a certain pin high and low. To
achieve a specific hertz frequency, the timer can use varying prescaler and
compare registry values. For most timer uses one would set a suitable prescaler
and change only the timer compare registry values, but not every tone can be
timed perfectly with any particular prescaler. Thus, both settings are changed
for every tone generated.

First we have to take into consideration that the Arduino Uno has a
clock frequency of 16MHz.
$$CPU\_CLK = 16 \cdot 10^6$$

To come up with the right prescaler we simply calculate the interval with a
given prescaler and check if it fits in a byte, the size of OCR.

$$OCR = prescaler \cdot \frac{CPU\_CLK}{freq \cdot 2}$$

Note that the frequency is doubled due to the fact that the timer needs to
alternate the pin output twice, namely high and low.

If the frequency is not suitable for any kind of prescaler it will generate
a tone as high as possible.

The prescaler of the timer can differ from tone to tone, thus making it
unsuitable for reuse.

\newpage
\subsection{Memory Management}
\label{sec:music-memory}
When inserting a song on the Arduino you will quickly find out that it
simply does not fit. Therefore we came up with the following compression
strategy.

We have analysed the song and concluded that it uses six distinct rhythm
symbols and thirteen distinct tones. If we enumerate these things
seperately, then the tones can fit in 4 bits (16 possibilities) and the rhythm
symbols in 3 bits (8 possibilities). Therefore we can compress a tone and
duration into a single byte while leaving one bit unused. Even though it's
a rather simple solution, it's still effective enough to compress the entire
song into 239 bytes.


The encoding could, if required later, be extended to use the unused bit.
The bit might then, for example, act as a flag to indicate that the remaining 7
bits are to be interpreted as an index into an existing array of notes. This
will tremendously decrease the size when having many repeated sequences in
the song.
