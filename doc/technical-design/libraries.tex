\section{Libraries}
\label{sec:libraries}

Instead of depending on the Arduino framework, the game makes use of the
\emph{Pleasant Uno AVR}\cite{pleasant-uno-avr} libraries. These libraries were
written in part to support this game, for a number of reasons. First of all,
the game is written in pure C. Because the Arduino framework is written in C++,
it would introduce unwanted \emph{C++-isms}. Secondly, the Arduino framework
and dependent libraries are rather heavy, using much more program memory than
desired. This is especially true for the default library to control the
MI0283QT-9 display that is used.

Pleasant Uno AVR offers a few different libraries, the ones of which that are
used by the game are discussed in the rest of this section.

\subsection{Pleasant Timer}
\label{sec:libraries/pleasant-timer}

Pleasant Timer offers a few functions and preprocessor aliases to make working
with the AVR's timers more \emph{pleasant}. Instead of requiring the programmer
to fiddle with individual bits in registers, it offers the functions
\texttt{timer0\_init}, \texttt{timer1\_init} and \texttt{timer2\_init} to set
up any desired configuration.

As an example, the game initializes timer 0 as follows. Without Pleasant Timer
this would involve several manipulations of the \texttt{TCCR0A},
\texttt{TCCR0B} and \texttt{TIMSK0} registers.

\begin{lstlisting}[language=C,frame=single]
  timer0_init(TIMER_WAVE_TYPE_NORMAL,
              TIMER_DEFAULT_WRAP_TYPE,
              TIMER_CLOCK_SOURCE_DIV_1024,
              TIMER_INTERRUPT_COMPARE_B,
              TIMER_DEFAULT_COMPARE_OUTPUT_MODE,
              TIMER_COMPARE_OUTPUT_MODE_TOGGLE);
\end{lstlisting}

\subsection{Pleasant USART}
\label{sec:libraries/pleasant-usart}

\begin{sloppypar}
  Pleasant USART provides functionality to control the USART module in the
  AVR\@. It provides the function \texttt{usart\_init} to set the baud rate,
  the \emph{asynchronous mode}, the use or non-use of a parity bit, the number
  of stop bits to use and the bit-size of each character. After initialization
  the functions \texttt{usart\_write}, \texttt{usart\_read},
  \texttt{usart\_byte\_available}, \texttt{usart\_write\_bytes},
  \texttt{usart\_read\_bytes}, \texttt{usart\_write\_string}, and
  \texttt{usart\_read\_string} can be used to transmit data.
\end{sloppypar}

As an example, the game initializes the USART module as follows. This results
in a baud rate of 115200, the use of \emph{double speed asynchronous mode}, no
parity bit, 1 stop bit and a character size of 8 bits.

\begin{lstlisting}[language=C,frame=single]
  usart_init(115200,
             USART_DEFAULT_ASYNCHRONOUS_MODE,
             USART_DEFAULT_PARITY,
             USART_DEFAULT_STOP_BIT_COUNT,
             USART_DEFAULT_CHARACTER_SIZE);
\end{lstlisting}

\subsection{Pleasant TWI}
\label{sec:libraries/pleasant-twi}

Pleasant TWI enables communication using the I²C protocol. It is used by the
game to receive input from a Nunchuck controller (see
Section~\ref{sec:nunchuck}).

\begin{sloppypar}
  Pleasant TWI provides the functions \texttt{twi\_init}, \texttt{twi\_write}
  and \texttt{twi\_read} for master operation, which is the only mode used by
  the game. Slave operation is also possible using the function
  \texttt{twi\_set\_address}, after which the callback functions
  \texttt{twi\_slave\_receive\_callback} and
  \texttt{twi\_slave\_transmit\_callback} are called when the master requests
  or sends data. The function \texttt{twi\_transmit\_reply} can be used to
  respond to requests.
\end{sloppypar}

\subsection{Pleasant LCD}
\label{sec:libraries/pleasant-lcd}

Pleasant LCD implements support for interaction with the MI0283QT-9 TFT display
with touch panel. It uses the \emph{Pleasant SPI} library for communication
with the display and the touch component, and \emph{Pleasant Timer} to set up
timer 1 to drive the brightness of the display.

Initialization of the LCD display is done using the function
\texttt{lcd\_init}, which expects an SPI clock speed to use. One should
normally use \texttt{LCD\_DEFAULT\_SPI\_CLOCK\_SPEED}, which is equivalent to
\texttt{SPI\_CLOCK\_SPEED\_DIV\_2}.

The brightness of the display can be set using the function
\texttt{lcd\_set\_brightness}, which modifies the comparator register of timer
1. In the game this function is called by \texttt{update\_lcd\_brightness} (see
Section~\ref{sec:potentiometer}).

\begin{sloppypar}
  Drawing to the display can be done using the functions
  \texttt{lcd\_draw\_pixel}, \texttt{lcd\_fill\_screen} and
  \texttt{lcd\_fill\_rect}. Each one of those functions fills (an area of) the
  screen with one color. If a larger area should be filled with multiple
  colors, drawing individual pixels using \texttt{lcd\_draw\_pixel} introduces
  lots of overhead. For each pixel it has to send commands to the display to
  select the area of the pixel, just to then send a single color to fill it
  with.
\end{sloppypar}

\begin{sloppypar}
  To more efficiently fill areas of the screen with multiple colors, the
  functions \texttt{lcd\_batch\_start}, \texttt{lcd\_batch\_draw} and
  \texttt{lcd\_batch\_stop} are provided. \texttt{lcd\_batch\_start} selects
  the area of the screen to fill, \texttt{lcd\_batch\_draw} fills individual
  pixels of the selected area, and \texttt{lcd\_batch\_stop} finishes the
  drawing operation. Pixels are filled from left to right, top to bottom. Note
  that which direction is \emph{left} or \emph{top} depends on the orientation
  of the display, which can be set using the function
  \texttt{lcd\_set\_orientation}.
\end{sloppypar}

Reading input from the touch panel is done using the function
\texttt{lcd\_touch\_read}. To improve the accuracy of the coordinates reported,
the touch panel needs to be \emph{calibrated}. This can be done using the
function \texttt{lcd\_touch\_start\_calibration}.
