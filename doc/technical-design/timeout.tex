\section{Timeouts}

Since we are not using the Arduino library, we want an equivalent of the
function \texttt{delay(int timeout)}, but with one difference: The delay
function may not hang. Therefore we use a flag which indicates if the given
time has elapsed. This will result in a small inaccuracy, because you have to
manually check if it has elapsed, but it makes it possible to have multiple
timeouts at the same time. This functionality is provided through the following
function: \texttt{set\_timeout(bool *flag, int timeout)}

If you want the exact same functionality as \texttt{delay}, you can achieve it
as follows:

\begin{lstlisting}[language=C,frame=single,captionpos=b,caption={A typical
delay function}]
/* Set a timeout of 10 seconds and let the program hang */
bool time_has_elapsed = false;
set_timeout(&time_has_elapsed, 61 * 10);
while(!time_has_elapsed);
\end{lstlisting}

You shouldn't use this when playing the game, because it stops the main
\emph{thread} of execution. However, you could use it when trying to connect
two devices to eachother. For example:

\begin{lstlisting}[language=C,frame=single,captionpos=b,caption={A typical
timeout function}]
/* Try to connect with another host device.
 * Unless it takes too long to make a connection (5s) */
bool connection_timeout = false;
set_timeout(&connection_timeout, 61 * 5);
ir_connect();
while(!ir_connected())
  if (connection_timeout) break;

/* If not already connected,
 * the device will host a connection itself */
if (!ir_connected()) {
  ir_host();
  while (!ir_connected());
}
\end{lstlisting}


A heap is used to manage the different timeouts. The timer interrupt will
decrease all timeouts in the heap (time complexity $O(n)$) and check for each
timeout if the time has elapsed. If it has, its boolean value will be set to
true. Because popping the value from the heap can be expensive, this is not
done in the interrupt routine. Instead the function
\texttt{clear\_elapsed\_timeouts()} should be called from the \emph{main
  thread}, which will remove passed timeouts from the heap.

Because the heap is modified from two different \emph{threads} of execution, a
\emph{race condition} may occur. For example, if the timer interrupt fires
while the heap is being reordered by the expensive pop function, it sees
incoherent data. To fix the issue, a \emph{lock} is used. While the heap is
being reordered, a variable is set to \texttt{true}. If the interrupt fires
during this time, it will instead increment a separate counter variable. This
variable represents the number of times the interrupt couldn't do its job.
Then, when the lock becomes available again, all timeouts will be decremented
by that amount. This corrects the missed ticks, but does introduce a slightly
less accurate timeout.
