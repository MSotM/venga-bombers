\section{Concepts}
\label{sec:concepts}

\subsection{World}
\label{sec:world}

The game is played on a grid. That grid is contained in the \emph{world}, which
is an object of type \texttt{world\_t}. It contains a simple array, containing
a \texttt{tile\_t} (See Section~\ref{sec:tiles}) for every location in the
grid. This array has a size of $\mathtt{WORLD\_WIDTH \times WORLD\_HEIGHT}$. A
pointer to each tile in the grid can be retrieved using the function
\texttt{world\_tile}, which takes \texttt{x} and \texttt{y} coordinates.

Before the start of the game, the world must be allocated and initialized using
the function \texttt{init\_world}. Then, on every iteration of the game loop,
the function \texttt{update\_world} will cause all objects in the world to
update.

\subsection{Tiles}
\label{sec:tiles}

Tiles are objects of type \texttt{tile\_t}, which is a type alias for
\texttt{uint8\_t}. The different attributes of the tile are encoded in those 8
bits. The bits represent the following attributes, also marked in appropriate
\emph{mask} constants:

\begin{itemize}
\item Two bits, as marked in \texttt{TILE\_MASK\_TYPE}, specifying the tile's
  \emph{tile type}. It defines whether the tile can be walked on and whether
  explosions affect it. It should be one of the following values, of type
  \texttt{tile\_type\_t}:

  \begin{itemize}
  \item \texttt{TILE\_TYPE\_EMPTY} for tiles that can be walked on;
  \item \texttt{TILE\_TYPE\_SOLID} for tiles that are solid, but transition to
    \texttt{TILE\_TYPE\_EMPTY} when hit by an explosion;
  \item \texttt{TILE\_TYPE\_STATIC} for tiles that are solid and that are not
    affected by explosions.
  \end{itemize}

  The value of these bits can be read using the function \texttt{tile\_type}
  and modified using the function \texttt{tile\_set\_type}.

\item Two bits, as marked in \texttt{TILE\_MASK\_UPGRADE}, specifying the
  tile's \emph{upgrade}. It defines what, if any, upgrade is on that tile. It
  should be one of the following values, of type \texttt{tile\_upgrade\_t}:

  \begin{itemize}
  \item \texttt{TILE\_UPGRADE\_NONE} for tiles without upgrades;
  \item \texttt{TILE\_UPGRADE\_RANGE} for tiles containing an upgrade that
    increases the player's bombs' explosion range;
  \item \texttt{TILE\_UPGRADE\_BOMBS} for tiles containing an upgrade that
    increases the amount of bombs the player can place;
  \item \texttt{TILE\_UPGRADE\_SPEED} for tiles containing an upgrade that
    increases the player's walking speed.
  \end{itemize}

  The value of these bits can be read using the function \texttt{tile\_upgrade}
  and modified using the function \texttt{tile\_set\_upgrade}.

\item One bit, as marked in \texttt{TILE\_MASK\_CONTAINS\_BOMB}, specifying
  whether or not there is a bomb on the tile. It gets set when a player places
  a bomb on that tile, and cleared when that bomb explodes. Additional
  information about the bomb is stored separately. See Section~\ref{sec:bombs}.

  The value of this bit can be read using the function
  \texttt{tile\_contains\_bomb} and modified using the function
  \texttt{tile\_set\_contains\_bomb}.

\item One bit, as marked in \texttt{TILE\_MASK\_CONTAINS\_EXPLOSION},
  specifying whether or not there is an explosion on the tile. It gets set when
  a bomb's explosion hits the tile, and cleared when the explosion disappears
  again. Additional information about the explosion is stored separately. See
  Section~\ref{sec:explosions}.

  The value of this bit can be read using the function
  \texttt{tile\_contains\_explosion} and modified using the function
  \texttt{tile\_set\_contains\_explosion}.

\item Two unused bits available for later use.
\end{itemize}

\subsection{Players}
\label{sec:players}

Players, of type \texttt{player\_t}, are user-controlled entities that move
around the world. They can place bombs, pick up upgrades, and get damaged by
explosions. Every player is, at any point in the game, on a certain tile, but
as you can see in Section~\ref{sec:tiles}, a tile is never marked as containing
a player. All interaction between a player and the world is initiated from the
side of the player, so checking whether or not a player is on the tile does not
need to be tested efficiently.

A player has a number of lives. When the tile a player is on is currently
affected by an explosion, one of its lives will be removed and the
\texttt{damage\_countdown} property of the player will be set. Until
\texttt{damage\_countdown} reaches 0, the player will not be damaged.
Furthermore, the \texttt{PLAYER\_FLAG\_SCORE\_UPDATED} flag is set on the
player. This causes the renderer to re-render the health of the player.

A player can move into any empty tile, except for ones with a bomb on them.
When the player moves, its \texttt{movement\_countdown} is reset to its
\texttt{movement\_default\_countdown}. Until
\texttt{movement\_countdown} reaches 0, the player cannot move.

When a player walks onto a tile containing an upgrade (See
Section~\ref{sec:tiles}), that upgrade gets picked up. What happens depends on
the type of the upgrade:

\begin{itemize}
\item An upgrade of type \texttt{TILE\_UPGRADE\_RANGE} causes the player's
  \texttt{explosion\_range} to be incremented. See Section~\ref{sec:bombs}.
\item An upgrade of type \texttt{TILE\_UPGRADE\_BOMBS} causes the player's
  \texttt{max\_bomb\_quantity} to be incremented.
\item An upgrade of type \texttt{TILE\_UPGRADE\_SPEED} causes the player's
  \texttt{movement\_default\_countdown} to be decremented. The
  \texttt{movement\_default\_countdown} will not be decremented to a value
  lower than \texttt{PLAYER\_MIN\_DEFAULT\_MOVEMENT\_COUNTDOWN}.
\end{itemize}

Every player has an ID, which is a number between 1 and \texttt{PLAYER\_COUNT}.
A pointer to the player with a certain ID can be retrieved using the function
\texttt{get\_player}.

There are two players in the game, as specified by \texttt{PLAYER\_COUNT}, and
those players are initialized using the function \texttt{init\_players}. The
\texttt{init\_players} function is called by \texttt{init\_world}, so it does
not usually need to be called from anywhere else. The function
\texttt{update\_world} (See Section~\ref{sec:world}) calls
\texttt{update\_players}, which updates the relevant properties of each player.

When an event (See Section~\ref{sec:events}) is generated to move a player, the
function \texttt{handle\_events} calls the function \texttt{player\_move}. This
function will return \texttt{true} if the player actually moved, or
\texttt{false} if that, for some reason, wasn't possible.

\subsection{Bombs}
\label{sec:bombs}

Players can place bombs on tiles. Doing so sets the
\texttt{tile\_contains\_bomb} bit, but more information needs to be stored for
each bomb. An array is maintained containing \texttt{BOMB\_COUNT} pre-allocated
\texttt{bomb\_t} instances, which will be used to store this additional
information.

Because all \texttt{bomb\_t} instances are pre-allocated, there needs to be
some way to check if such a bomb is active. That way is by checking if the
bomb's \texttt{player} property is not \texttt{NULL}. When a bomb is placed,
this property becomes a pointer to the \texttt{player\_t} that placed it, and
when a bomb explodes it is reset to \texttt{NULL}.

Bombs are placed using the function \texttt{place\_bomb}. It takes a
\texttt{player\_t} instance and places the bomb at that player's location. On
every game loop iteration, the function \texttt{update\_world} (See
Section~\ref{sec:world}) calls \texttt{update\_bombs} in order to update all
bombs' countdowns.

When a bomb's countdown reaches 0, the function \texttt{trigger\_bomb} will
spawn a number of \emph{explosions}. These explosions will be spawned in
straight horizontal and vertical lines, and the amount spawned in each
direction depends on the player's \texttt{explosion\_range}. When a
\emph{static} tile is encountered in one direction, the spreading of explosions
in that direction stops. When a \emph{solid} tile is encountered, an explosion
is placed on that tile, the tile is made \emph{empty}, and further spreading in
that direction is halted.

\subsection{Explosions}
\label{sec:explosions}

Spawning an explosion sets the \texttt{tile\_contains\_explosion} bit of the
tile the explosion is on, but some more information needs to be stored for each
explosion. An array is maintained containing \texttt{EXPLOSION\_COUNT}
pre-allocated \texttt{explosion\_t} instances, which will be used to store this
additional information.

An \texttt{explosion\_t} instance is very simple. It only contains a position
and a countdown. A single tile can never have multiple active explosions on it.
Instead, were that to happen, the function \texttt{activate\_explosion} re-uses
the previous \texttt{explosion\_t} instance active on that tile.

Explosions, aside from changing the \texttt{tile\_contains\_explosion} bit of
the tile they're on, don't directly affect the world. Instead, the function
\texttt{trigger\_bomb} (See Section~\ref{sec:bombs}) clears any \emph{solid}
tiles it places an explosion on, and players (See Section~\ref{sec:players})
determine whether they've been hit by an explosion through the
\texttt{tile\_contains\_explosion} bit (See Section~\ref{sec:tiles}).

On every game loop iteration, the function \texttt{update\_world} (See
Section~\ref{sec:world}) calls the function \texttt{update\_explosions} in
order to update all explosion countdowns. When such a countdown reaches 0, the
\texttt{tile\_contains\_explosion} bit of the tile it is on is cleared, and the
explosion is no longer considered active.

\subsection{Events}
\label{sec:events}

Updates to the world should be initiated using \emph{events}. An \emph{event
  queue} is maintained, containing instances of \texttt{event\_t}. When a user
provides input, an event should be added to that queue using the function
\texttt{queue\_event}. Events in the queue are handled by the function
\texttt{handle\_events}, which calls the appropriate functions to update the
world (See Section~\ref{sec:world}).

The event queue is a pre-allocated array of \texttt{event\_t} instances, which
will be re-used constantly. The function \texttt{handle\_events} retrieves
events from the queue using \texttt{dequeue\_event}. After it is done
processing the event, it calls \texttt{free\_event} to reset its values.

An event contains an \emph{event type} and the \emph{ID} of the player it
affects. See Section~\ref{sec:players}. Both of these need to be provided to
\texttt{queue\_event}. The event type needs to be a value of type
\texttt{event\_type\_t}, which can be either of the following:

\begin{itemize}
\item \texttt{EVENT\_TYPE\_MOVE\_UP} to move the player one tile up;
\item \texttt{EVENT\_TYPE\_MOVE\_RIGHT} to move the player one tile to the
  right;
\item \texttt{EVENT\_TYPE\_MOVE\_DOWN} to move the player one tile down;
\item \texttt{EVENT\_TYPE\_MOVE\_LEFT} to move the player one tile to the left;
\item \texttt{EVENT\_TYPE\_PLACE\_BOMB} to make the player place a bomb at its
  current location.
\end{itemize}

A last value of \texttt{event\_type\_t} is \texttt{EVENT\_TYPE\_PROCESSED},
which is used by the queue to indicate that the event is inactive. It should
never be passed to \texttt{queue\_event}.

\subsection{Score}
\label{sec:score}

Every player has a certain score, which indicates how well the player is
performing in the game. When a player performs certain actions or is affected
by those of others, the score for that player will change. This change of score
is implemented using the function \texttt{player\_add\_score}. When the score
is changed, the \texttt{PLAYER\_FLAG\_SCORE\_UPDATED} flag is set on the
player. This causes the renderer to perform a re-render of the players score.

At the end of the game, each player's score is added to the list of highscores,
on condition that it's higher than another score currently in the list.
Highscores are stored using the type \texttt{highscores\_t}, which is a
\emph{typedef} for an array of 16-bit unsigned integers, of length
\texttt{HIGHSCORE\_COUNT} (5).

The current list of highscores can be retrieved using the function
\texttt{highscores\_read}. This is primarily used by the highscores screen.
At the end of a game, the function \texttt{highscores\_add} is called with the
score of each player. It will take care of inserting the score at the
appropriate place in the highscores array, shifting lower scores down.

Highscores are saved in \emph{EEPROM} memory, at a specific location (43). The
byte before the highscores data is set to a \emph{magic} value, 42. If that
byte is not equal to the magic value, it is assumed that EEPROM does not
contain highscores. Reading from EEPROM is done using the function
\texttt{eeprom\_read\_block}, to read the entire array into memory at once.
Writing is similarly done using \texttt{eeprom\_write\_block}. These functions
are provided by the AVR standard library, in the file \texttt{avr/eeprom.h}.
